\ifx\pdfminorversion\undefined\else\pdfminorversion=4\fi
\documentclass[aspectratio=169,t]{beamer}
%\documentclass[aspectratio=169,t,handout]{beamer}

% English version FAU Logo
\usepackage[english]{babel}
% German version FAU Logo
%\usepackage[ngerman]{babel}

\usepackage[utf8]{inputenc}
\usepackage[T1]{fontenc}
\usepackage{amsmath,amssymb}
\usepackage{graphicx}
\usepackage{listings}
\usepackage{url}
\usepackage{enumitem}
\usepackage{hyperref}
\usepackage{fontawesome}
\usepackage{graphicx}
\usepackage{booktabs}
\usepackage{tikz}
\usepackage{tikz-cd}

\tikzset{
    vertex/.style = {
        circle,
        fill            = black,
        outer sep = 2pt,
        inner sep = 1pt,
    }
}
\usetikzlibrary{matrix,mindmap}
\usetikzlibrary{arrows,decorations.pathmorphing,backgrounds,fit,positioning,shapes.symbols,chains,intersections}
\tikzset{level 1/.append style={sibling angle=50,level distance = 165mm}}
\tikzset{level 2/.append style={sibling angle=20,level distance = 45mm}}
\tikzset{every node/.append style={scale=1}}
\newcommand{\tikzmark}[1]{\tikz[remember picture] \node[coordinate] (#1) {#1};}
% Options:
%  - inst:      Institute
%                 med:      MedFak FAU theme
%                 nat:      NatFak FAU theme
%                 phil:     PhilFak FAU theme
%                 rw:       RWFak FAU theme
%                 rw-jura:  RWFak FB Jura FAU theme
%                 rw-wiso:  RWFak FB WISO FAU theme
%                 tf:       TechFak FAU theme
%  - image:     Cover image on title page
%  - plain:     Plain title page
%  - longtitle: Title page layout for long title
\usetheme[%
  image,%
  longtitle,%
  tf
]{fau}

% Enable semi-transparent animation preview
\setbeamercovered{transparent}


\lstset{%
  language=Python,
  tabsize=2,
  basicstyle=\tt,
  keywordstyle=\color{blue},
  commentstyle=\color{green!50!black},
  stringstyle=\color{red},
  numbers=left,
  numbersep=0.5em,
  xleftmargin=1em,
  numberstyle=\tt
}


% Title, authors, and date
\title[KDD]{Chapter II: Data}
\subtitle{Knowledge Discovery in Databases}
\author[L.~Melodia]{Luciano Melodia M.A.}
% English version
\institute[Department]{Evolutionary Data Management, Friedrich-Alexander University Erlangen-Nürnberg}
% German version
%\institute[Lehrstuhl]{Lehrstuhl, Friedrich-Alexander-Universit\"at Erlangen-N\"urnberg}
\date{Summer semester 2021}
% Set additional logo (overwrites FAU seal)
%\logo{\includegraphics[width=.15\textwidth]{themefau/art/xxx/xxx.pdf}}


\begin{document}
  % Title
  \maketitle

  { 
    \setbeamertemplate{footline}{}
    \begin{frame}{Chapter II: Getting to know your data}
    This is our agenda for this lecture:
        \begin{itemize}
            \item \textbf{Data objects and attribute types}
            \item Basic statistical descriptions of data
            \item Data visualization
            \item Measuring data similarity and dissimilarity
            \item Summary
        \end{itemize}
    \end{frame}
  }

  { 
    \setbeamertemplate{footline}{}
    \begin{frame}{Types of data sets}
      \begin{columns}
        \begin{column}{0.45\textwidth}
          \textbf{Records:}
          \begin{itemize}[noitemsep]
              \item Relational records.
              \item Data matrix, e.g. numerical matrix, crosstabs.
              \item Document data: text documents, \textbf{term-frequency vectors}. \tikzmark{n1}
              \item \textbf{Transaction data}. \tikzmark{n2}
          \end{itemize}
          \textbf{Graph and network:}
          \begin{itemize}[noitemsep]
              \item World wide web.
              \item Social of information networks.
              \item Molecular structures.
          \end{itemize}
        \end{column}
        \begin{column}{0.45\textwidth}  %%<--- here
        \begin{table}
         \begin{tabular}{|c|c|c|c|c|c|c|}
            \multicolumn{1}{c|}{} & \rotatebox[origin=c]{270}{team} & \rotatebox[origin=c]{270}{couch} & \rotatebox[origin=c]{270}{play} & \rotatebox[origin=c]{270}{ball} & \rotatebox[origin=c]{270}{score} & \rotatebox[origin=c]{270}{game} \\ \hline
            \tikzmark{t1} Document1 & 3 & 0 & 5 & 0 & 2 & 6 \\ \hline
            Document2 & 0 & 7 & 0 & 2 & 1 & 0 \\ \hline
            Document3 & 0 & 1 & 0 & 0 & 1 & 2 \\
            \hline
          \end{tabular}\\[0.5cm]
          \begin{tabular} { | c | l |}
          \hline
          \textbf{TID} & \textbf{Items} \\
          \hline
          \tikzmark{t2} 1 & Bread, Coke, Milk\\
          2 & Beer, Bread\\
          3 & Beer, Coke, Diapers, Milk\\
          4 & Beer, Bread, Diapers, Milk\\
          5 & Coke, Diapers, Milk \\
          \hline
          \end{tabular}
        \end{table}
        \begin{tikzpicture}[remember picture,overlay]
           \path[draw=blue,thick,->]<1-> ([yshift=1mm]n1) -- (t1);
           \path[draw=blue,thick,->]<1-> ([yshift=1mm]n2) -- (t2);
        \end{tikzpicture}
        \end{column}
      \end{columns}
    \end{frame}
  }

  { 
    \setbeamertemplate{footline}{}
    \begin{frame}{Types of data sets}
          \textbf{Ordered data:}
          \begin{itemize}[noitemsep]
              \item Video data: sequences of images.
              \item Temporal data: time series.
              \item Sequential data: transaction sequences.
              \item Genetic sequence data.
          \end{itemize}
          \textbf{Spatial, image and multimedia:}
          \begin{itemize}[noitemsep]
              \item Spatial data: maps.
              \item Image data.
              \item Video data.
          \end{itemize}
    \end{frame}
  }

  { 
    \setbeamertemplate{footline}{}
    \begin{frame}{Important characteristics of structured data}
        \textbf{Dimensionality}:\\
        Curse of dimensionality (sparse high-dimensional data spaces).\\[0.2cm]
        % make the example with the volume of a cube

        \textbf{Sparsity}:\\
        Only presence counts.\\[0.2cm]

        \textbf{Resolution}:\\
        Patterns depend on the scale.\\[0.2cm]

        \textbf{Distribution}:\\
        Centrality and dispersion.
    \end{frame}
  }

  { 
    \setbeamertemplate{footline}{}
    \begin{frame}{Data objects}
      \textbf{Data sets are made up of data objects}.\\
      \textbf{A data object represents an entity}.\\[0.2cm]

      Examples:
      \begin{itemize}
          \item Sales database: customers, store items, sales.
          \item Medical database: patients, treatments.
          \item University database: students, professors, courses.
      \end{itemize}

      They are also called:\\
      Sampels, examples, instances, data points, objects, tuples, \ldots\\[0.2cm]

      \textbf{Data objects are described by attributes}:
      \begin{itemize}
          \item Database rows $\rightarrow$ data objects.
          \item Columns $\rightarrow$ attributes.
      \end{itemize}
    \end{frame}
  }

  { 
    \setbeamertemplate{footline}{}
    \begin{frame}{Attributes}
    \textbf{Attribute}:\\
    Sometimes also in other context: field, dimension, feature, variable, \ldots\\[0.2cm]
    A data field encodes the property of an entity or feature of a data object.\\
    E.g. \texttt{customer\_ID}, \texttt{name}, \texttt{address}.\\[0.5cm]

    \textbf{Types}:
    \begin{itemize}
      \item Nominal.
      \item Binary.
      \item Ordinal.
      \item Numerical:
      \begin{itemize}
        \item Interval scaled.
        \item Ratio scaled.
      \end{itemize}
    \end{itemize}
    \end{frame}
  }

  { % Questions?
    \setbeamertemplate{footline}{}
    \begin{frame}[c]
      \begin{center}
        Thank you for your attention.\\
        {\bf Any questions about the second chapter?}\\[0.5cm]
        Ask them now, or again, drop me a line: \\ 
        \faSendO \ \texttt{luciano.melodia@fau.de}.
      \end{center}
    \end{frame}
  }
\end{document}

