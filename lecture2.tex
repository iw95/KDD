\ifx\pdfminorversion\undefined\else\pdfminorversion=4\fi
\documentclass[aspectratio=169,t]{beamer}
%\documentclass[aspectratio=169,t,handout]{beamer}

% English version FAU Logo
\usepackage[english]{babel}
% German version FAU Logo
%\usepackage[ngerman]{babel}

\usepackage[utf8]{inputenc}
\usepackage[T1]{fontenc}
\usepackage{amsmath,amssymb}
\usepackage{graphicx}
\usepackage{listings}
\usepackage{url}
\usepackage{hyperref}
\usepackage{fontawesome}
\usepackage{tikz-cd}

% Options:
%  - inst:      Institute
%                 med:      MedFak FAU theme
%                 nat:      NatFak FAU theme
%                 phil:     PhilFak FAU theme
%                 rw:       RWFak FAU theme
%                 rw-jura:  RWFak FB Jura FAU theme
%                 rw-wiso:  RWFak FB WISO FAU theme
%                 tf:       TechFak FAU theme
%  - image:     Cover image on title page
%  - plain:     Plain title page
%  - longtitle: Title page layout for long title
\usetheme[%
  image,%
  longtitle%
]{fau}

% Enable semi-transparent animation preview
\setbeamercovered{transparent}


\lstset{%
  language=C++,
  tabsize=2,
  basicstyle=\tt,
  keywordstyle=\color{blue},
  commentstyle=\color{green!50!black},
  stringstyle=\color{red},
  numbers=left,
  numbersep=0.5em,
  xleftmargin=1em,
  numberstyle=\tt
}


% Title, authors, and date
\title[KDD]{Chapter I: Introduction}
\subtitle{Knowledge Discovery in Databases}
\author[L.~Melodia]{Luciano Melodia M.A.}
% English version
\institute[Department]{Evolutionary Data Management, Friedrich-Alexander University Erlangen-Nürnberg}
% German version
%\institute[Lehrstuhl]{Lehrstuhl, Friedrich-Alexander-Universit\"at Erlangen-N\"urnberg}
\date{Summer semester 2021}
% Set additional logo (overwrites FAU seal)
%\logo{\includegraphics[width=.15\textwidth]{themefau/art/xxx/xxx.pdf}}


\begin{document}
  % Title
  \maketitle

  { 
    \setbeamertemplate{footline}{}
    \begin{frame}{Chapter I: Introduction}
    This is our agenda for this lecture:
        \begin{itemize}
            \item \textbf{Why data mining?}
            \item What is data mining?
            \item A multi-dimensional view of data mining.
            \item What kinds of data can be mined?
            \item What kinds of patterns can be mined?
            \item What technologies are used?
            \item What kinds of applications are targeted?
            \item Major issues in data mining.
            \item A brief history of data mining.
            \item Summary.
        \end{itemize}
    \end{frame}
  }

  { 
    \setbeamertemplate{footline}{}
    \begin{frame}{Why data mining?}
    \textbf{The explosive growth of data: from terabytes to petabytes and more.}\\
        \begin{itemize}
            \item Data collection and availability:
                \begin{itemize}
                    \item Automated data collection tools.
                    \item Database systems.
                    \item World wide web.
                    \item Computerized society.
                    \item Digitization.
                \end{itemize}
            \item Major sources of abundant data:
                \begin{itemize}
                    \item Business: web, e-commerce, transactions, stocks \ldots
                    \item Science: remote sensing, bioinformatics, scientific simulation \ldots
                    \item Society: news, digital cameras, social media \ldots
                \end{itemize}
            \item The era of \textbf{big data} (as inflationary used buzzword).
        \end{itemize}
    \textbf{We are drowning in data, but starving for knowledge.} \textbf{Necessity is the mother of invention.}\\
    For data mining it is the automated analysis of massive data sets.
    \end{frame}
  }

  { 
    \setbeamertemplate{footline}{}
    \begin{frame}{Evolution of sciences I}
        \begin{itemize}
            \item Before $1600$, era of \textbf{empirical science}.
            \item $1600-1950$s, rise of \textbf{theoretical science}.
                  \begin{itemize}
                      \item Each discipline has grown a theoretical component.
                      \item Theoretical models often motivate experiments and generalize our understanding.
                  \end{itemize}
            \item $1950-1990$s, rise of \textbf{computational science}.
                  \begin{itemize}
                      \item Over the last $50$ years most disciplines have grown a third, computational branch.
                      \begin{itemize}
                          \item E.g. empirical, theoretical and computational ecology.
                          \item E.g. physics, linguistics or biology.
                      \end{itemize}
                  \end{itemize}
            \item Computational science traditionally meant simulation.
            \item It grew out of our inability to describe reality by closed-form mathematical models.
        \end{itemize}
    \end{frame}
  }

  { 
    \setbeamertemplate{footline}{}
    \begin{frame}{Evolution of sciences II}
        \begin{itemize}
            \item $1990-$now, rise of \textbf{data science}.
                  \begin{itemize}
                      \item The flood of data from new instruments and modern simulations.
                      \item The ability to economically store and manage petabytes of data.
                      \item The internet makes all these archives world wide accessible.
                      \item Scientific \emph{information management}, \\
                            acquisition,\\
                            organization, \\
                            query and \\
                            visualization scale almost linearly with amount of data.
                      \item \textbf{Data mining} is a major new challenge!
                  \end{itemize}
        \end{itemize}
    \end{frame}
  }

  { % Questions?
    \setbeamertemplate{footline}{}
    \begin{frame}[c]
      \begin{center}
        Thank you for your attention.\\
        {\bf Any questions about orga?}\\[0.5cm]
        Ask them now, or again, drop me a line: \\ 
        \faSendO \ \texttt{luciano.melodia@fau.de}.
      \end{center}
    \end{frame}
  }
\end{document}

