\ifx\pdfminorversion\undefined\else\pdfminorversion=4\fi
\documentclass[aspectratio=169,t]{beamer}
%\documentclass[aspectratio=169,t,handout]{beamer}

% English version FAU Logo
\usepackage[english]{babel}
% German version FAU Logo
%\usepackage[ngerman]{babel}

\usepackage[utf8]{inputenc}
\usepackage[T1]{fontenc}
\usepackage{amsmath,amssymb}
\usepackage{graphicx}
\usepackage{listings}
\usepackage{url}
\usepackage{hyperref}
\usepackage{fontawesome}
% Options:
%  - inst:      Institute
%                 med:      MedFak FAU theme
%                 nat:      NatFak FAU theme
%                 phil:     PhilFak FAU theme
%                 rw:       RWFak FAU theme
%                 rw-jura:  RWFak FB Jura FAU theme
%                 rw-wiso:  RWFak FB WISO FAU theme
%                 tf:       TechFak FAU theme
%  - image:     Cover image on title page
%  - plain:     Plain title page
%  - longtitle: Title page layout for long title
\usetheme[%
  image,%
  longtitle%
]{fau}

% Enable semi-transparent animation preview
\setbeamercovered{transparent}


\lstset{%
  language=C++,
  tabsize=2,
  basicstyle=\tt,
  keywordstyle=\color{blue},
  commentstyle=\color{green!50!black},
  stringstyle=\color{red},
  numbers=left,
  numbersep=0.5em,
  xleftmargin=1em,
  numberstyle=\tt
}

% Title, authors, and date
\title[KDD]{Knowledge Discovery in Databases}
\subtitle{With a view towards machine learning}
\author[L.~Melodia]{Luciano Melodia M.A.}
% English version
\institute[Department]{Evolutionary Data Management, Friedrich-Alexander University Erlangen-Nürnberg}
% German version
%\institute[Lehrstuhl]{Lehrstuhl, Friedrich-Alexander-Universit\"at Erlangen-N\"urnberg}
\date{Summer semester 2021}
% Set additional logo (overwrites FAU seal)
%\logo{\includegraphics[width=.15\textwidth]{themefau/art/xxx/xxx.pdf}}


\begin{document}
  % Title
  \maketitle

  { % Motivation
    \setbeamertemplate{footline}{}
    \begin{frame}{Who and for whom?}
      \begin{itemize}
        \item Luciano Melodia M.A.
            \begin{itemize}
              \item Evolutionary Data Management.
              \item \texttt{luciano.melodia@fau.de}.
            \end{itemize}
        \item Master course in Computer Science:
            \begin{itemize}
              \item For partial fulfillment of a module with \underline{$5$ ECTS}.
              \item Examination: Oral exam together with another lecture, such as:
                  \begin{itemize}
                      \item DBRN (Datenbanken in Rechnernetzen).
                      \item DSS (Datenstromsysteme, German only).
                      \item DW (Data Warehousing).
                      \item TAS (Transaktionssysteme, also in Englisch).
                  \end{itemize}
              \item Other programs:
                  \begin{itemize}
                      \item M.Sc. International Information Systems.
                      \item M.Sc. Computational Engineering.
                      \item M.Sc. Mechanical Engineering.
                      \item M.Sc. Information- and Communication Technology.
                      \item \ldots
                  \end{itemize}
            \end{itemize}
      \end{itemize}
    \end{frame}
  }

  { % Outline
    \setbeamertemplate{footline}{}
    \begin{frame}{Prerequisites}
      \begin{itemize}
          \item IDB (Implementierung von Datenbanksystemen, every WS).
          \item IDS (Introduction to Database Systems)
              \begin{itemize}
                  \item With parts on implementation:
                    \begin{itemize}
                        \item Index structures.
                        \item Query processing.
                        \item \ldots
                    \end{itemize}
              \end{itemize}
      \end{itemize}
    \end{frame}
  }

  { % Lecture
    \setbeamertemplate{footline}{}
    \begin{frame}{Lecture}
      \begin{itemize}
          \item Time and place (during \texttt{SARS-CoV2} pandemic via Zoom):
              \begin{itemize}
                  \item Tuesday, 8:15 - 9:45 c.t.
                  \item Room: K$2$-$119$.
                  \item Zoom-room: \href{https://fau.zoom.us/j/2164886110}{https://fau.zoom.us/j/2164886110}.
              \end{itemize}
          \item Lecture notes will be available as \texttt{PDF} via StudOn.
          \item Additional information is also available in StudOn.
          \item What about \textbf{feedback, questions, remarks} or \textbf{complaints}?
              \begin{itemize}
                  \item Always welcome.
                  \item Drop me a line: \texttt{luciano.melodia@fau.de}.
                  \item Consultation hours: Thursday, 5-6 p.m.
                  \item Room: 08.132.
                  \item Course will be evaluated electronically.
              \end{itemize}
      \end{itemize}
    \end{frame}
  }


  { % Lecture
    \setbeamertemplate{footline}{}
    \begin{frame}{Lecture}
      \begin{itemize}
          \item The lecture is based on the following book:
              \begin{itemize}
                  \item Jiawei Han, Micheline Kamber, and Jian Pei: \emph{Data Mining – Concepts and Technologies}, 3rd ed. Waltham, M.A.: Morgan Kaufmann, 2012 (The Morgan Kaufmann Series in Data Management Systems).
                  \item \small{Copies are available at the TNZB.}
                  \item \small{Slides have been downloaded from \href{http://hanj.cs.illinois.edu/}{Jiawei Han's web server} and modified.}
                  \item \small{Originally these slides were created by Prof. Dr.-Ing. Klaus Meyer-Wegener.}
              \end{itemize}
          \item Also interesting and related textbooks are:
              \begin{itemize}
                  \item Hongbo Du: \emph{Data Mining Techniques and Applications}, 1st ed. Cengage Learning, 2010.
                  \item Ian Witten, Eibe Frank, and Mark Hall: \emph{Data Mining: Practical Machine Learning Tools and Techniques}, 3rd ed. Burlington, M.A.: Morgan Kaufmann, 2011 (The Morgan Kaufmann Series in Data Management Systems).
                  \item \ldots
              \end{itemize}
      \end{itemize}
    \end{frame}
  }

  { % Goal
    \setbeamertemplate{footline}{}
    \begin{frame}{Our goal}
      \begin{itemize}
          \item Introduce you to the principles of data mining. \\ This is the core of knowledge discovery in databases.
          \item We'll make you familiar with:
              \begin{itemize}
                  \item Some KDD pipelines.
                  \item The challenge of data mining on large sets of data.
                  \item Technologies available for data analysis.
                  \item Systems offering these technologies.
                  \item Some applications \ldots
              \end{itemize}
      \end{itemize}
    \end{frame}
  }

  { % Goal
    \setbeamertemplate{footline}{}
    \begin{frame}{Companion lecture: \textbf{Deep Learning}}
      \begin{itemize}
          \item First time hold in winter term $2017/2018$,\\
                now offered again at:
                \begin{itemize}
                    \item Thursday, 8-9:45 a.m. c.t.
                    \item Room: H4.
                    \item Pattern Recognition Lab.
                    \item Lecture is in Englisch.
                \end{itemize}
          \item Topics covered:
                \begin{itemize}
                    \item Classification using Neural Networks, \\
                          probably in much more detail.
                    \item No databases, no indexes, no clustering \ldots
                \end{itemize}
          \item This lecture gives a broader overview, thus is less detailed.
      \end{itemize}
    \end{frame}
  }

  { % Modules
    \setbeamertemplate{footline}{}
    \begin{frame}{Modules}
      \begin{itemize}
          \item A module is the \textbf{unit of examination} at FAU and must have at least $5$ ECTS.
          \item Thus, KDD alone is a too small lecture.
                \begin{itemize}
                    \item Just $2.5$ ECTS, which corresponds to \\
                          $75$ hours of work, \\
                          $30$ for attendance and \\
                          $45$ hours for own studies and exam preparation.
                \end{itemize}
          \item \textbf{Combine KDD with another lecture} of the same size, such as:
                \begin{itemize}
                    \item DBRN (Datenbanken in Rechnernetzen).
                    \item DSS (Datenstromsysteme).
                    \item DW (Data Warehousing).
                    \item TAS or TRASYS (Transaktionssysteme).
                \end{itemize}
          \item Examination: a single oral exam for \textbf{both lectures} together.\\
                $30$ minutes with approximately $15$ minutes per lecture.
      \end{itemize}
    \end{frame}
  }

  { % Modules
    \setbeamertemplate{footline}{}
    \begin{frame}{Exceptions for modules}
      \begin{itemize}
          \item Combinations with DW (Data Warehousing):
                \begin{itemize}
                    \item DW has its own written exam.
                    \item Thus, you'll need a supplementary oral exam for KDD alone.\\
                          This exam will be also a $30$ minutes exam.
                \end{itemize}
          \item \textbf{Medical Engineering} Master program:
                \begin{itemize}
                    \item Modules of $2.5$ ECTS are still allowed.
                    \item KDD can be chosen in \\
                          \emph{Studienrichtung Medizinische Bild- und Datenverarbeitung \\ M5 Medizintechnische Vertiefungsmodule (BDV).}\\
                          Examination number 392229.
                    \item Examination: an oral exam for $30$ minutes.
                \end{itemize}
      \end{itemize}
    \end{frame}
  }

  { % Exam
    \setbeamertemplate{footline}{}
    \begin{frame}{Exam}
      \begin{itemize}
          \item \textbf{Registration} for examination:
                \begin{itemize}
                    \item At the examination office (dt. Prüfungsamt) or rather in \texttt{MeinCampus}.
                    \item Period (probably): May $15$ -- Mai $30, 2021$ (three weeks).
                    \item Please denote your examination number!
                \end{itemize}
          \item \textbf{Appointment} for oral exams:
                \begin{itemize}
                    \item In StudOn, or visit our \href{https://www.cs6.tf.fau.de/lehre/pruefungsinformationen/}{homepage}: Lehrstuhl $\rightarrow$ Lehre $\rightarrow$ Prüfungsinformationen.
                    \item Or visit \url{https://www.cs6.tf.fau.de/lehre/pruefungsinformationen/}.
                \end{itemize}
      \end{itemize}
    \end{frame}
  }

  { % Questions?
    \setbeamertemplate{footline}{}
    \begin{frame}[c]
      \begin{center}
        Thank you for your attention.\\
        {\bf Any questions about orga?}\\[0.5cm]
        Ask them now, or again, drop me a line: \\
        \faSendO \ \texttt{luciano.melodia@fau.de}.
      \end{center}
    \end{frame}
  }
\end{document}
